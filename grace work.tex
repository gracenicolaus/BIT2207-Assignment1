\documentclass{article}
\begin{document}
\title{THE HUMAN RIGHTS PROBLEM IN UGANDA {}}
\author{Kyalisiima Nicholas}
\maketitle
\begin{abstract}
The three most serious human rights problems in the country include: lack of respect for the integrity of the person ( abuse of suspects and detainees) restrictions on civil liberties (freedom of assembly, expression, 
the media, and association); this is very evident in the country as journalists and reporters are roughed up and beaten by the police ; and violence and discrimination against marginalized groups, such as women (sexual and gender
-based violence), children (sexual abuse and ritual killing), persons with disabilities.

 Other human rights problems include harsh prison conditions, arbitrary and politically motivated arrest and detention, lengthy pretrial detention, restrictions on the right to a fair trial, official corruption, or mob violence, trafficking in persons, and child labor.

Although the government occasionally takes steps to punish officials who commit abuses, whether in the security services or elsewhere, impunity is a problem.

\end{abstract}
\subsection{ Some of the common problems that exist}
\begin{description}
\item[1.	Lack of respect for the integrity of the person. ]
\begin{itemize}
 \item[]  
 \item 
These include the abuse of suspects and detainees. This is evident in the where suspects and detainees have on numerous occasions claimed to have been tortured by police for information and also during their arrests. During arrests, suspects are roughed up by police and at times are even beaten to a point whereby they bleed and during transportation to the prison cell, suspects are roughly dragged and thrown onto police pickup trucks. For example during the strikes at Makerere university, police seriously beats students, even those who may not be participating in the strike and on some occasions like in 2008, students lose their lives as a result of stray bullets fired by the police. This is a clear example of lack of respect for the integrity of the person and also a clear violation of the right to life. That is to say that the security forces in Uganda have total disregard for the integrity of human beings when they are doing their job and this is what makes them violate the rights of the Ugandan citizens.
 \end{itemize}
\item[2.	Restrictions on civil liberties including; Freedom of speech and press.]
\begin{itemize}
 \item[] 
\item
The constitution and the law provide for freedom and press, but the government restricted these rights. Freedom of speech and expression; security forces and government officials occasionally interrogate and detain radio and television presenters and political leaders who make political statements l of the government and use slander laws and national security as a ground to restrict freedom of speech. This is further seen through the arrest of the radio and television presenters, journalists such as Payira of top radio. Unlike in the previous year there were no reports the government deployed officials to monitor public meetings in schools to prevent students from holding debates about a successor to the president.
Press and media freedom, the police and security platforms monitored all the radio and media communications during the 2016 presidential elections .this was against the human rights of citizens in Uganda.

\end{itemize}
\item[3.	Violence and discrimination on groups of people such as women,children and many more]
\begin{itemize}
\item []
\item
The law prohibits discrimination based on race, sex, religion among others. The government did not enforce laws against discrimination adequately and locally. Thus affecting many people in the country .
Women and children. Rape and domestic violence .the law criminalizes rape which is punished by life imprisonment or death penalty. The law also criminalizes domestic violence and provides up to two years imprisonment for conviction . Other harmful actions against women and children include sexual harassment, reproductive rights such as forced marriages .discrimination during job employment opportunities

\end{itemize}
\end{description}  
\subsection{SOLUTIONS TO THE PROBLEM AFFECTING HUMAN RIGHTS }
As for the above mentioned problems, solutions have been put into action in order to  solve them. 
\begin{itemize}
\item For the lack of respect for integrity of the persons, laws have been set up to help in maintaining law and order in society. With this people live in harmony and peace .The government should also enforce strict laws of the authorities that misuse their rights.
\item The government should help people know their laws throughout the country in order for the people to know their right. With this in place, people are able to report to the police where need be. Hence promoting human rights in Uganda. This will help on the problem of restrictions on civil liberties.
\item For the case of discrimination ,violence and human abuse .people should create organizations like human rights platforms that help them put out their views on a country wide level .this will help people get knowledge of their rights and legal standing in the society in which they live.
\end{itemize}
\subsection{ Conclusion}
The human rights problem in Uganda should be taken more seriously because even the government which is responsible for the protection of the rights of the citizens plays a major role in the violation of human rights. Therefore in order to effectively address this issue of human rights in Uganda, the government must first re-evaluate itself and its members before it moves forward to fight against acts of human rights abuse carried out by the citizens of Uganda.  
\end{document}